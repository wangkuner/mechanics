\documentclass[12pt]{article}
    % fonts
    \usepackage{ctex}
    \usepackage{fontspec}%set fonts
    \newfontfamily\cnew{Courier New}
    \usepackage{amsmath}
    \usepackage{newtxmath}
    \setmainfont{Times New Roman}

    % paper set
    \usepackage{geometry}%set paper
    \geometry{
        a4paper,
        total={170mm,257mm},
        left=20mm,
        top=20mm,
    }
    % set paragraph format
    %\setlength{\baselineskip}{1.0em}
    \renewcommand{\baselinestretch}{1.0}
    % \usepackage{booktabs} % for three-line table
    \usepackage{array}
    \usepackage[numbers]{gbt7714}
    \numberwithin{equation}{section}

    \title{Classical Mechanics}
    \author{Kun Wang}
    \date{\today}

    \begin{document}
        \maketitle
        \section{牛顿力学}
        \begin{equation}
            F_k=\frac{dp_k}{dt}
        \end{equation}

        \section{Lagrange力学}
        \subsection{基本概念\cite{2006shen}}
        位形空间:位形是质点系各质点或连续体中各小单元的位置或位移的集合。位形坐标系所在空间就是位形空间。\par
        约束:约束分类有理想约束和非理想约束,完整约束/非完整约束,稳定约束/不稳定约束。\\
        理想约束:在任何虚位移上,约束反力的元功之和为零的约束,反之约束反力元功之和不为零的约束为非理想约束。\\
        稳定约束:约束方程中不显含时间$t$的约束,反之显含时间$t$的约束为不稳定约束。\\
        完整约束:约束方程中不含速度或者速度可积分消掉的约束,约束方程中含有不可积分速度的为非完整约束。
        \par
        虚功原理:
        \subsection{第一类Lagrange方程}
	Lagrange乘子法:$3n$个自由度系统,$s$个约束,选择Lagrange乘子,使得$s$个不独立的虚位移前括号中的项为零,
	剩余的$3n-s$个独立的虚位移前括号中的项也都等于零。
        \subsection{第二类Lagrange方程}
        \begin{equation}
            \frac{\partial}{\partial q_k}\left(\frac{d}{dt}\right)=\frac{d}{dt}\left(\frac{\partial}{\partial q_k}\right)
        \end{equation}

	    \subsubsection{Newton方程推导Lagrange方程}
	    \subsubsection{d'Alembert方程推导Lagrange方程}

        \subsubsection{Newton方程和d'Alembert方程的区别}
        Newton方程中的力是所有作用力,包括主动力和约束力。d'Alembert方程中只含有主动力。
        \begin{equation}
            L=T-V
        \end{equation}

        \begin{equation}
            \frac{d}{dt}\left(\frac{\partial L}{\partial \dot{q}_k}\right)-\frac{\partial L}{\partial q_k}=Q_k
        \end{equation}

        \subsection{Lagrange方程首次积分}

        \subsubsection{广义动量积分}
        $L$与$q_k$无关,则有
        \begin{equation}
            \frac{\partial L}{\partial q_k}=0
        \end{equation}
        \begin{equation}
            \frac{d}{dt}\frac{\partial L}{\partial \dot{q}_k}=0
        \end{equation}
        \begin{equation}
            p_k=\frac{\partial L}{\partial \dot{q}_k}=const
        \end{equation}

        \subsubsection{广义能量积分}

        \noindent 1) 稳定约束下的Hamilton积分\\
        $L$与$t$无关,则有
        \begin{equation}
            \begin{aligned}
                \frac{dL}{dt}&=\frac{\partial L}{\partial q_k}\dot{q}_k+\frac{\partial L}{\partial q_k}\ddot{q}_k
                =\frac{\partial L}{\partial q_k}\dot{q}_k+\frac{d}{dt}\left( \frac{\partial L}{\partial \dot{q}_k}\dot{q}_k \right)
                -\frac{d}{dt}\frac{\partial L}{\partial q_k}\dot{q}_k\\
                &=\frac{d}{dt}\left( \frac{\partial L}{\partial \dot{q}_k}\dot{q}_k \right)
                -\left( \frac{d}{dt}\frac{\partial L}{\partial q_k}-\frac{\partial L}{\partial q_k} \right)\dot{q}_k
                =\frac{d}{dt}\left( \frac{\partial L}{\partial \dot{q}_k}\dot{q}_k \right)
            \end{aligned}
        \end{equation}
        \begin{equation}
           \frac{d}{dt}\left( \dot{q}_k\frac{\partial L}{\partial \dot{q}_k}-L \right)=0 
        \end{equation}
        \begin{equation}
           H(q-k,\dot{q}_k,t)=\dot{q}_k\frac{\partial L}{\partial \dot{q}_k}-L=const 
        \end{equation}

        \subsection{变分问题Euler方程}
        \subsubsection{Euler-Lagrange方程}
        \begin{equation}
            J=\int_A^Bf(x,y,\dot{y})dx
        \end{equation}
        求上式的极值。\\
        变分法基本预备定理:如果函数$f(x)$在域$[x_1,x_2]$上连续且对于只满足某些一般条件1)一阶或若干阶可微分;
        2)在域$[x_1,x_2]$的端点处为0;3)$\|\delta y(x)\|<\epsilon$或
        $\|\delta y(x)\|$和$\|\delta y'(x)<\epsilon\|$的任意函数$\delta y(x)$,有
        $$\int_{x_1}^{x_2}f(x)\delta y(x)dx=0$$
        则在域$[x_1,x_2]$上有$f(x)=0$。

        \noindent 泛函取极值的必要条件是泛函的变分等于0。\\
        微分变分可对易:$\delta(dx)=d(\delta x)$

        \begin{equation*}
            \begin{aligned}
                \Delta J&=\int_A^Bf(x,y+\delta y,(y+\delta y)')dx-\int_A^Bf(x,y,\dot{y})dx\\
                &=\int_A^Bf(x,y+\delta y,\dot{y}+\delta \dot{y})dx-\int_A^Bf(x,y,\dot{y})dx
            \end{aligned}
        \end{equation*}
        将上式按泰勒公式展开有$\Delta J=\delta J+\delta^2J+\cdots$\\
        一阶变分$\delta J=\int_A^B\left(\frac{\partial f}{\partial y}\delta y +\frac{\partial f}{\partial \dot{y}}\delta\dot{y}\right)dx$\\
        二阶变分$\delta^2J=\int_A^B \left( \frac{\partial^2f}{\partial y^2}\delta y^2+
        2\frac{\partial^2f}{\partial y\partial\dot{y}}\delta y\delta\dot{y}+
        \frac{\partial^2f}{\partial\dot{y}^2}\delta\dot{y}^2 \right)dx$\\
        使用分部积分法(integration by parts)将$\delta\dot{y}$转变为$\delta y$有
        \begin{equation*}
            \begin{aligned}
                \delta J&=\int_A^B\left(\frac{\partial f}{\partial y}\delta y +\frac{\partial f}{\partial \dot{y}}\delta\dot{y}\right)dx\\
                &=\int_A^B\frac{\partial f}{\partial y}\delta ydx+\frac{\partial f}{\partial \dot{y}}\delta y\Big|_A^B-
                \int_A^B\frac{d}{dx}\left(\frac{\partial f}{\partial \dot{y}}\right)\delta ydx
            \end{aligned}
        \end{equation*}
        由于$\delta y$在$A,B$处取0,故有
        \begin{equation*}
            \begin{aligned}
                \delta J&=\int_A^B\frac{\partial f}{\partial y}\delta ydx-
                \int_A^B\frac{d}{dx}\left(\frac{\partial f}{\partial \dot{y}}\right)\delta ydx\\
                &=\int_A^B\left(\frac{\partial f}{\partial y}-\frac{d}{dx}\left(\frac{\partial f}{\partial \dot{y}}\right)\right)\delta ydx
            \end{aligned}
        \end{equation*}
        根据泛函取极值得必要条件$\delta J=0$有
        \begin{equation}
            \delta J=\int_A^B\left(\frac{\partial f}{\partial y}-\frac{d}{dx}\left(\frac{\partial f}{\partial \dot{y}}\right)\right)\delta ydx=0
        \end{equation}
        根据变分法基本预备定理有
        \begin{equation}\label{E-L}
            \frac{\partial f}{\partial y}-\frac{d}{dx}\left(\frac{\partial f}{\partial \dot{y}}\right)=0
        \end{equation}
        上式便是变分原理的Euler-Lagrange方程。
        
        \subsubsection{含多个未知函数泛函的变分}
        \begin{equation}
            J=\int_A^Bf(x,y_1,y_2,\cdots,\dot{y}_1,\dot{y}_2,\cdots)dx
        \end{equation}

        \begin{equation*}
            \delta J=\int_A^B\left(\frac{\partial f}{\partial y_i}\delta y_i-\frac{\partial f}{\partial \dot{y}_i}\delta \dot{y}_i\right)dx=0
        \end{equation*}
        使用分部积分可得
        \begin{equation}\label{mE-L}
            \frac{\partial f}{\partial y_i}-\frac{d}{dx}\left(\frac{\partial f}{\partial \dot{y}_i}\right)=0
        \end{equation}
        记$\mathbf{y}=[y_1,y_2,\cdots,y_n]^T$,$\mathbf{\dot{y}}=[\dot{y}_1,\dot{y}_2,\cdots,\dot{y}_n]^T$。上式写成向量的形式为
        \begin{equation}\label{vE-L}
            \frac{\partial f}{\partial \mathbf{y}}-\frac{d}{dx}\left(\frac{\partial f}{\partial \dot{\mathbf{y}}}\right)=\mathbf{0}
        \end{equation}
        \subsection{Hamilton原理}
        \noindent Lagrangian $L=T-U=L(t,\mathbf{q},\dot{\mathbf{q}})$\\
        Hamilton函数$H=\int_{t_1}^{t_2}Ldt$\\
        Hamilton原理(Hamilton principle):$\delta H=0$
        \begin{equation}\label{vLagrange}
            \frac{\partial L}{\partial \mathbf{q}}-\frac{d}{dt}\left(\frac{\partial L}{\partial \dot{\mathbf{q}}}\right)=\mathbf{0}
        \end{equation}
        % \bibliographystyle{plain}
        \bibliography{cm}
    \end{document}
