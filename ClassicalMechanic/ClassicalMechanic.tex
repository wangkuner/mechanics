\documentclass[12pt]{article}
    % fonts
    \usepackage{ctex}
    \usepackage{fontspec}%set fonts
    \newfontfamily\cnew{Courier New}
    \usepackage{amsmath}
    \usepackage{newtxmath}
    \setmainfont{Times New Roman}

    % paper set
    \usepackage{geometry}%set paper
    \geometry{
        a4paper,
        total={170mm,257mm},
        left=20mm,
        top=20mm,
    }
    % set paragraph format
    %\setlength{\baselineskip}{1.0em}
    \renewcommand{\baselinestretch}{1.0}
    % \usepackage{booktabs} % for three-line table
    \usepackage{array}
    \usepackage[numbers]{gbt7714}
    \numberwithin{equation}{section}

    \title{Classical Mechanics}
    \author{Kun Wang}
    \date{\today}

    \begin{document}
        \maketitle
        \section{牛顿力学}
        \begin{equation}
            m\ddot{x}=F
        \end{equation}

        \section{Lagrange力学}
        \subsection{基本概念\cite{2006shen}}
        位形空间:位形是质点系各质点或连续体中各小单元的位置或位移的集合。位形坐标系所在空间就是位形空间。\par
        约束:约束分类有理想约束和非理想约束,完整约束/非完整约束,稳定约束/不稳定约束。\\
        理想约束:在任何虚位移上,约束反力的元功之和为零的约束,反之约束反力元功之和不为零的约束为非理想约束。\\
        稳定约束:约束方程中不显含时间$t$的约束,反之显含时间$t$的约束为不稳定约束。\\
        完整约束:约束方程中不含速度或者速度可积分消掉的约束,约束方程中含有不可积分速度的为非完整约束。
        \par
        虚功原理:
        \subsection{第一类Lagrange方程}
        \subsection{第二类Lagrange方程}
        \begin{equation}
            \frac{\partial}{\partial q_k}\left(\frac{d}{dt}\right)=\frac{d}{dt}\left(\frac{\partial}{\partial q_k}\right)
        \end{equation}

        \begin{equation}
            L=T-V
        \end{equation}

        \begin{equation}
            \frac{d}{dt}\left(\frac{\partial L}{\partial \dot{q}_k}\right)-\frac{\partial L}{\partial q_k}=Q_k
        \end{equation}

        % \bibliographystyle{plain}
        \bibliography{cm}
    \end{document}