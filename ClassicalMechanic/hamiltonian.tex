
\subsection{Hamilton Equations}

\noindent\textbf{Legendre transformation}

\noindent Single variables

% \begin{definition}
    The Legendre transformation of the convex function $f(x)$ is another function $g(p)$ defined as 
    $g(p)=\max_{x}F(p,x)$, where $F(p,x)=px-f(x)$.
% \end{definition}

The function $F(p,x)$ gets the maximum with respect to $x$ only when $\partial F/\partial x =0$ 
which means $p = f'(x)$. So, the transformation of the function $x$ is 
\begin{equation}
        g(p) = px-f(x),\ p=f'(x)
\end{equation}

The variable $x$ and $p$ are dual variables. The functions $f$ and $g$ are the Legendre transformation mutually and 
the function $g(p)$ is also a convex function.

\noindent Multi-variables

For the multi-variable function $f(\mathbf{x}):\mathbf{x}\in \mathbb{R}^{n}\rightarrow \mathbb{R}$, 
its Legendre transformation is defined as 
\begin{equation}
        g(\mathbf{p}) = \mathbf{p}^{T}\mathbf{x}-f(\mathbf{x}),\ \mathbf{p}=f'(\mathbf{x})
\end{equation}

\noindent \textbf{Hamiltonian}
The Legendre transformation of the Lagrangian $L(\mathbf{q},\dot{\mathbf{q}},t)$ is  
\begin{equation}
    H(\mathbf{p},\mathbf{q},t)=\mathbf{p}^{T}\dot{\mathbf{q}}-L(\mathbf{q},\dot{\mathbf{q}},t),\ \mathbf{p}=\frac{\partial L}{\partial \dot{\mathbf{q}}}
\end{equation}

$H(\mathbf{p},\mathbf{q},t)$ is the so-called Hamiltonian. 
The Lagrange equation is equivalent to the following Hamilton equations 
\begin{equation}
    \begin{aligned}
        &\dot{\mathbf{p}} = -\frac{\partial H}{\partial \mathbf{q}} \\
        &\dot{\mathbf{q}} = \frac{\partial H}{\partial \mathbf{q}}
    \end{aligned}
\end{equation}

Proof:
% \begin{proof}
    The total differential of $H$ is 
    $$dH = \frac{\partial H}{\partial \mathbf{p}}d\mathbf{p} + \frac{\partial H}{\partial \mathbf{q}}d\mathbf{q} + \frac{\partial H}{\partial t}dt.$$
    The total differential of $\mathbf{p}^{T}\dot{\mathbf{q}}-L(\mathbf{q},\dot{\mathbf{q}},t)$ is 
    $$dH = \dot{\mathbf{p}}^{T}d\mathbf{p} - \frac{\partial L}{\partial \mathbf{q}}d\mathbf{q} - \frac{\partial L}{\partial t}dt$$
    The two equations are equivalent, so 
\begin{equation}
    \dot{\mathbf{q}}=\frac{\partial H}{\partial \mathbf{p}},\ \frac{\partial H}{\partial \mathbf{q}}=-\frac{\partial L}{\partial \mathbf{q}},\ \frac{\partial H}{\partial t}=-\frac{\partial H}{\partial t}.
\end{equation}
    Combining with the Lagrange equation and $\mathbf{p}=\frac{\partial L}{\partial \dot{\mathbf{q}}}$, one can obtain the Hamilton equations.
% \end{proof}

The Euler's homogeneous function theorem for quadric form $f(\mathbf{x})$ is 
\begin{equation}
    \mathbf{x}^{T}\frac{\partial f}{\partial \mathbf{x}}=2f
\end{equation}

Assuming the kinematic energy $T$ is a quadric form with respect to $\dot{\mathbf{q}}$, one can obtain 
\begin{equation}
    H(\mathbf{p},\mathbf{q},t)=\mathbf{p}^{T}\dot{\mathbf{q}}-L(\mathbf{q},\dot{\mathbf{q}},t)=\mathbf{p}^{T}\dot{\mathbf{q}}-T+U=\dot{\mathbf{q}}^{T}\frac{\partial T}{\partial \dot{\mathbf{q}}}-T+U=T+U.
\end{equation}
Therefore, the Hamiltonian is the total energy.
