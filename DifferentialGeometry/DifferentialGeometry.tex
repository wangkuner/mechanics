\documentclass[12pt]{article}
    % fonts
    \usepackage{ctex}
    % \usepackage{mathptmx}
    \usepackage{amsmath}
    % \usepackage[notext]{stix}
    % \usepackage{newtxmath}
    \usepackage{unicode-math}
    \usepackage{fontspec}%set fonts
    \setmathfont{STIX2Math.otf}
    \setmainfont{Times New Roman}
    \newfontfamily\cnew{Courier New}

    % \theoremstyle{definition}
    \newtheorem{definition}{定义}[section]
    \newtheorem{theorem}{定理}[section]
    % paper set
    \usepackage{geometry}%set paper
    \geometry{
        a4paper,
        total={170mm,257mm},
        left=20mm,
        top=20mm,
    }
    % set paragraph format
    %\setlength{\baselineskip}{1.0em}
    \renewcommand{\baselinestretch}{1.0}
    % \usepackage{booktabs} % for three-line table
    \usepackage{array}
    \usepackage[colorlinks,
            linkcolor=cyan,
            anchorcolor=blue,
            citecolor=blue]{hyperref}
    \usepackage{bookmark}
    \usepackage{xcolor}

    \title{Learning Notes of \textit{Differential Geometry}}
    \author{Kun Wang}
    % \date{2021年4月26日-\today}

    \begin{document}
        \maketitle

        \section{Curved surface}

\textbf{正则参数曲面}的定义:
\begin{equation}
    \mathbf{r}\left(u,\, v\right) = \left( \begin{array}{c}
        x(u,\,v) \\ y(u,\,v) \\ z(u,\,v)
    \end{array} \right)
\end{equation}
满足:
\begin{enumerate}
    \item[1)] $x(u,\,v)$, $y(u,\,v)$, $z(u,\,v)$有3次以上连续偏导数;
    \item[2)] $\mathbf{r}_{,u}\times\mathbf{r}_{,v}\neq\mathbf{0}$。
\end{enumerate}

\textbf{容许参数变换}:$u=u(\tilde{u},\, \tilde{v})$, $v=v(\tilde{u},\, \tilde{v})$, 满足
\begin{enumerate}
    \item[1)] $u=u(\tilde{u},\, \tilde{v})$, $v=v(\tilde{u},\, \tilde{v})$有3次以上连续偏导数;
    \item[2)] $\left| \frac{\partial\left(u,\, v\right)}{\partial\left(\tilde{u},\, \tilde{v}\right)} \right|\neq 0$。
\end{enumerate}

其中$\mathbf{r}_{,u}\times\mathbf{r}_{,v}$指向曲面的正侧,
$\left| \frac{\partial\left(u,\, v\right)}{\partial\left(\tilde{u},\, \tilde{v}\right)} \right|>0$
为保持参数曲面定向不变的充要条件。

\textbf{正则曲面}定义:
\begin{definition}
    $S$是$E^{3}$中一张正则曲面,则$S \subseteq E^{3}$, $\forall p\in S$, 
    $\exists$点$p$的邻域$V\in E^{3}$以及$U\subseteq E^{2}$, 使得$\mathbf{r}:U\mapsto V\cap S$为双射,
    且$\mathbf{r}\left(u,\,v\right),\ (u,\,v)\in U$为正则参数曲面。
\end{definition}

\textbf{切向量}:曲面$S$上经过点$p$的任意一条连续可微曲线在该点的切向量为曲面$S$在点$p$的切向量。
曲纹坐标线的切向量为$\mathbf{r}_{,u}$, $\mathbf{r}_{,v}$. 
曲面$S$上过点$p$的曲线$\mathbf{r}\left(u(t),\,v(t)\right)$的切向量为
\begin{equation}
    \frac{\mathrm{d}\mathbf{r}}{\mathrm{d}t} = \mathbf{r}_{,u}\frac{\mathrm{d}u}{\mathrm{d}t} + 
    \mathbf{r}_{,v}\frac{\mathrm{d}v}{\mathrm{d}t}
\end{equation}
可以看出,点$p$处切向量是$\mathbf{r}_{,u}$和$\mathbf{r}_{,v}$的线性组合。
由于$\mathbf{r}_{,u}$, $\mathbf{r}_{,v}$是线性无关的,
故曲面$S$在点$p$的全体切向量构成一个二维向量空间,称为曲面$S$在点$p$的切空间,记作$T_{p}S$。
空间$E^{3}$中经过点$p$由切向量$\mathbf{r}_{,u}$, $\mathbf{r}_{,v}$张成的平面为曲面$S$在点$p$的\textbf{切平面}。
其参数方程为
\begin{equation}
    \mathbf{x}\left(\xi,\,\eta\right)=\mathbf{r}\left(u,\, v\right) + \xi\mathbf{r}_{,u} + \eta\mathbf{r}_{,v}
\end{equation}
\textbf{法向量}:
\begin{equation}
    \mathbf{n}\left(u,\,v\right)=\frac{\mathbf{r}_{,u}\times\mathbf{r}_{,v}}{\|\mathbf{r}_{,u}\times\mathbf{r}_{,v}\|}
\end{equation}
$E^{3}$中经过点$p$以法向量$\mathbf{n}\left(u,\,v\right)$为方向向量的直线为曲面$S$在点$p$的法线,其参数方程为
\begin{equation}
    \mathbf{x}\left(t\right)=\mathbf{r}\left(u,\, v\right) + t\mathbf{n}\left(u,\,v\right)
\end{equation}

$\{\mathbf{r}\left(u,\, v\right),\ \mathbf{r}_{,u},\ \mathbf{r}_{,v},\ \mathbf{n}\left(u,\,v\right)\}$构成曲面$S$上的\textbf{自然标架}。
一般来说$\mathbf{r}_{,u},\ \mathbf{r}_{,v}$不是正交向量。

第一基本形式

正则参数曲面$S$在点$\mathbf{r}\left(u,\,v\right)$处的任意一个切向量表示为
\begin{equation}\label{eq:first-diff}
    \mathrm{d}\mathbf{r} = \mathbf{r}_{,u}\mathrm{d}u + \mathbf{r}_{,v}\mathrm{d}v
\end{equation}
因为$\mathbf{r}_{,u},\ \mathbf{r}_{,v}$不一定是正交向量,故切向量的内积一般不能表示为切向量在自然基底下坐标分量的平方和。
如果知道自然基底的度量系数,切向量内积就可以表示为$\left(\mathrm{d}u,\,\mathrm{d}v\right)$的二次型。

\textbf{第一类基本量}
\begin{equation}
    \begin{aligned}
        E &= \mathbf{r}_{,u}\cdot\mathbf{r}_{,u} \\
        F &= \mathbf{r}_{,u}\cdot\mathbf{r}_{,v} = \mathbf{r}_{,v}\cdot\mathbf{r}_{,u} \\
        G &= \mathbf{r}_{,v}\cdot\mathbf{r}_{,v}
    \end{aligned}
\end{equation}

给定容许参数变换,$u=u(\tilde{u},\, \tilde{v})$, $v=v(\tilde{u},\, \tilde{v})$,
参数曲面的第一类基本量变为$\tilde{E},\ \tilde{F},\ \tilde{G}$,根据变换
\begin{equation}\label{eq:transform}
    \begin{aligned}
        \mathbf{r}_{,\tilde{u}} &= \mathbf{r}_{,u}\frac{\partial u}{\partial \tilde{u}} + \mathbf{r}_{,v}\frac{\partial v}{\partial \tilde{u}} \\
        \mathbf{r}_{,\tilde{v}} &= \mathbf{r}_{,u}\frac{\partial u}{\partial \tilde{v}} + \mathbf{r}_{,v}\frac{\partial v}{\partial \tilde{v}}
    \end{aligned}
\end{equation}
可得
\begin{equation}
    \begin{aligned}
        \tilde{E} &= \mathbf{r}_{,\tilde{u}}\cdot\mathbf{r}_{,\tilde{u}} = 
        \left(\frac{\partial u}{\partial \tilde{u}},\ \frac{\partial v}{\partial \tilde{u}}\right) \left(\begin{array}{cc}
        E & F \\ F & G
        \end{array}\right)\left(\begin{array}{c}
        \frac{\partial u}{\partial \tilde{u}} \\ \frac{\partial v}{\partial \tilde{u}}
    \end{array}\right) \\
        \tilde{F} &= \mathbf{r}_{,\tilde{u}}\cdot\mathbf{r}_{,\tilde{v}} = 
        \left(\frac{\partial u}{\partial \tilde{u}},\ \frac{\partial v}{\partial \tilde{u}}\right) \left(\begin{array}{cc}
        E & F \\ F & G
        \end{array}\right)\left(\begin{array}{c}
        \frac{\partial u}{\partial \tilde{v}} \\ \frac{\partial v}{\partial \tilde{v}}
    \end{array}\right) \\
        \tilde{F} &= \mathbf{r}_{,\tilde{v}}\cdot\mathbf{r}_{,\tilde{u}} = 
        \left(\frac{\partial u}{\partial \tilde{v}},\ \frac{\partial v}{\partial \tilde{v}}\right) \left(\begin{array}{cc}
        E & F \\ F & G
        \end{array}\right)\left(\begin{array}{c}
        \frac{\partial u}{\partial \tilde{v}} \\ \frac{\partial v}{\partial \tilde{v}}
    \end{array}\right)
    \end{aligned}
\end{equation}
即
\begin{equation}\label{eq:transform-matrix}
    \left(\begin{array}{cc} \tilde{E} & \tilde{F} \\ \tilde{F} & \tilde{G} \end{array}\right) = 
    \mathbf{J}^{T}\left(\begin{array}{cc} E & F \\ F & G \end{array}\right)\mathbf{J}
\end{equation}
其中
\begin{equation}
    \mathbf{J} = \left(\begin{array}{cc} \frac{\partial u}{\partial \tilde{u}} & \frac{\partial u}{\partial \tilde{v}} \\ 
        \frac{\partial v}{\partial \tilde{u}} & \frac{\partial v}{\partial \tilde{v}} \end{array}\right)
\end{equation}
即第一类基本量在容许参数变换下仅差一个合同变换。

切向量内积为
\begin{equation}
    I = \mathrm{d}\mathbf{r} \cdot \mathrm{d}\mathbf{d} = 
    \left(\mathrm{d}u,\ \mathrm{d}v\right)\left(\begin{array}{cc}
        E & F \\ F & G
    \end{array}\right)\left(\begin{array}{c}
        \mathrm{d}u \\ \mathrm{d}v
    \end{array}\right)
\end{equation}
二次微分式$I$称作曲面$S$的\textbf{第一基本形式},与曲面参数选取无关。因为根据式\eqref{eq:transform},
\begin{equation}
    \begin{aligned}
    \mathrm{d}\mathbf{r} &= \mathbf{r}_{,u}\mathrm{d}u + \mathbf{r}_{,v}\mathrm{d}v  \\
    &= \mathbf{r}_{,u}\left(\frac{\partial u}{\partial \tilde{u}}\mathrm{d}\tilde{u} + \frac{\partial u}{\partial \tilde{v}}\mathrm{d}\tilde{v}\right) + 
       \mathbf{r}_{,v}\left(\frac{\partial v}{\partial \tilde{u}}\mathrm{d}\tilde{u} + \frac{\partial v}{\partial \tilde{v}}\mathrm{d}\tilde{v}\right)  \\
    &= \left(\mathbf{r}_{,u}\frac{\partial u}{\partial \tilde{u}} + \mathbf{r}_{,v}\frac{\partial v}{\partial \tilde{u}}\right)\mathrm{d}\tilde{u} + 
       \left(\mathbf{r}_{,u}\frac{\partial u}{\partial \tilde{v}} + \mathbf{r}_{,v}\frac{\partial v}{\partial \tilde{v}}\right)\mathrm{d}\tilde{v} \\
    &= \mathbf{r}_{,\tilde{u}}\mathrm{d}\tilde{u} + \mathbf{r}_{,\tilde{v}}\mathrm{d}\tilde{v}
    \end{aligned}
\end{equation}
即一次微分式\eqref{eq:first-diff}与参数选取无关,故$I$为一次微分式的内积也与参数选取无关。

\textcolor{red}{\textbf{第一类基本量在容许参数变换下差一个合同变换,但是第一类基本形式却保持不变。}}

正则曲面上参数曲线的弧长可以用第一类基本量表示,给定参数曲线$\mathbf{r}\left(u(t),\, v(t)\right)$, 
曲线切向量为
\begin{equation}
    \frac{\mathrm{d}\mathbf{r}\left(u(t),\, v(t)\right)}{\mathrm{d}t} = 
    \mathbf{r}_{,u}\frac{\mathrm{d}u}{\mathrm{d}t} + \mathbf{r}_{,v}\frac{\mathrm{d}v}{\mathrm{d}t}
\end{equation}
曲线弧长为
\begin{equation}
    s=\int_{a}^{b}\|\mathbf{r}'\left(t\right)\|\mathrm{d}t = 
    \int_{a}^{b}\sqrt{E\left(\frac{\mathrm{d}u}{\mathrm{d}t}\right)^{2} + 
    2F\frac{\mathrm{d}u}{\mathrm{d}t}\frac{\mathrm{d}v}{\mathrm{d}t} + 
    G\left(\frac{\mathrm{d}v}{\mathrm{d}t}\right)}\mathrm{d}t
\end{equation}

在参数曲面上取曲线$u$, $u+\mathrm{d}u$, $v$, $v+\mathrm{d}v$围成的面积微元,其面积为
\begin{equation}
    \begin{aligned}
    \mathrm{d}\sigma &= \| \mathbf{r}_{,u}\times\mathbf{r}_{,v} \|\mathrm{d}u\mathrm{d}v
    = \|\mathbf{r}_{,u}\|\|\mathbf{r}_{,v}\|\sin\angle\left(\mathbf{r}_{,u},\,\mathbf{r}_{,v}\right)\mathrm{d}u\mathrm{d}v \\
    &= \sqrt{EG}\sqrt{1-\left(\frac{F}{\sqrt{EG}}\right)^{2}}\mathrm{d}u\mathrm{d}v = \sqrt{EG-F^{2}}\mathrm{d}u\mathrm{d}v
    \end{aligned}
\end{equation}
正则参数曲面面积:
\begin{equation}
    A=\iint_{D}\sqrt{EG-F^{2}}\mathrm{d}u\mathrm{d}v
\end{equation}
容许参数变换不改变正则参数曲面面积公式的形式,即
\begin{equation}
    A=\iint_{D}\sqrt{EG-F^{2}}\mathrm{d}u\mathrm{d}v = \iint_{D}\sqrt{\tilde{E}\tilde{G}-\tilde{F}^{2}}\mathrm{d}\tilde{u}\mathrm{d}\tilde{v}
\end{equation}
证明:给定容许参数变换$u=u(\tilde{u},\, \tilde{v})$, $v=v(\tilde{u},\, \tilde{v})$,
根据式\eqref{eq:transform-matrix}可得
\begin{equation*}
    \det\left(\begin{array}{cc} \tilde{E} & \tilde{F} \\ \tilde{F} & \tilde{G} \end{array}\right) = 
    \tilde{E}\tilde{G}-\tilde{F}^{2} =
    J^{2}\det\left(\begin{array}{cc} E & F \\ F & G \end{array}\right) = 
    J^{2}\left(EG-F^{2}\right)
\end{equation*}
即
\begin{equation*}
    \sqrt{\tilde{E}\tilde{G}-\tilde{F}^{2}} = J\sqrt{EG-F^{2}}
\end{equation*}
其中
\begin{equation*}
    J = \det\left(\mathbf{J}\right)
\end{equation*}
带入
\begin{equation*}
    A = \iint_{D}\sqrt{EG-F^{2}}J\mathrm{d}\tilde{u}\mathrm{d}\tilde{v}
\end{equation*}
得
\begin{equation*}
    A = \iint_{D}\sqrt{\tilde{E}\tilde{G}-\tilde{F}^{2}}\mathrm{d}\tilde{u}\mathrm{d}\tilde{v}
\end{equation*}
故参数曲面面积不随参数变换而变化。


正交参数曲线网

\begin{theorem}
    假定正则参数曲面$S:\mathbf{r}=\mathbf{r}\left(u,\, v\right)$上有两个\textbf{处处线性无关}的\textbf{连续可微}的切向量场,
    $\mathbf{a}\left(u,\,v\right)$, $\mathbf{b}\left(u,\,v\right)$, $\forall p\in S,\ \exists U_{p}\subseteq S$
    以及在在$U$上的新的参数系$\left(\tilde{u},\,\tilde{v}\right)$,
    使得新参数曲线的切向量$\mathbf{r}_{,\tilde{u}},\ \mathbf{r}_{,\tilde{v}}$分别与$\mathbf{a}$, $\mathbf{b}$平行,
    即$\mathbf{r}_{,\tilde{u}}//\mathbf{a},\ \mathbf{r}_{,\tilde{v}}//\mathbf{b}$。
\end{theorem}

\begin{theorem}
    给定正则参数曲面$S:\mathbf{r}=\mathbf{r}\left(u,\, v\right)$上一点$p$,
    必然存在一个点$p$的领域$U(p)\subseteq S$,以及新的参数$\left(\tilde{u},\,\tilde{v}\right)$,
    使得新参数曲线的切向量是正交的,即$\left(\tilde{u},\,\tilde{v}\right)$是曲面$S$在$U$上的正交参数系。
\end{theorem}
\textbf{证明}:正则参数曲面$\mathbf{r}=\mathbf{r}\left(u,\, v\right)$参数曲线的切向量作Schmidt正交化。
$$\mathbf{e}_{1}=\frac{\mathbf{r}_{,u}}{\|\mathbf{r}_{,u}\|}=\frac{1}{\sqrt{E}}\mathbf{r}_{,u}$$
$$\mathbf{b}=\mathbf{r}_{,v}+\lambda\mathbf{e}_{1}$$
$$\mathbf{b}\cdot\mathbf{e}_{1}=0$$
$$\lambda=-\mathbf{r}_{,v}\cdot\mathbf{r}_{,u}/\|\mathbf{r}_{,u}\|=-\frac{F}{\sqrt{E}}$$
$$\mathbf{b}=\mathbf{r}_{,v}-\frac{F}{E}\mathbf{r}_{,u}$$
$$\mathbf{e}_{2}=\frac{\mathbf{b}}{\|\mathbf{b}\|} = \frac{1}{\sqrt{EG-F^{2}}}\left(\sqrt{E}\mathbf{r}_{,v}-\frac{F}{\sqrt{E}}\mathbf{r}_{,u}\right)$$
得到曲面$S$正交向量场$\mathbf{e}_{1},\ \mathbf{e}_{2}$。
记
\begin{equation*}
    \left\{\begin{array}{l}
    \mathbf{e}_{1}=a_{1}(u,\,v)\mathbf{r}_{,u} + a_{2}(u,\,v)\mathbf{r}_{,v} \\
    \mathbf{e}_{2}=b_{1}(u,\,v)\mathbf{r}_{,u} + b_{2}(u,\,v)\mathbf{r}_{,v}
    \end{array}\right.
\end{equation*}
$a_{1}(u,\,v)$, $a_{2}(u,\,v)$, $b_{1}(u,\,v)$, $b_{2}(u,\,v)$均为连续可微函数,且
$a_{1}^{2}+a_{2}^{2}\neq 0$, $b_{1}^{2}+b_{2}^{2}\neq 0$,
$\left| \begin{array}{cc} a_{1} & a_{2} \\ b_{1} & b_{2} \end{array} \right|\neq 0$
因为根据引理,存在$\lambda(u,\,v)\neq 0$, $\mu(u,\, v)\neq 0$,使得
\begin{equation*}
    \begin{aligned}
    \mathrm{d}\tilde{u} &=\lambda\left(b_{2}\mathrm{d}u-b_{1}\mathrm{d}v\right)\\
    \mathrm{d}\tilde{v} &=\mu\left(-a_{2}\mathrm{d}u+a_{1}\mathrm{d}v\right)
    \end{aligned}
\end{equation*}
\begin{equation*}
    \left[\begin{array}{c}
        \mathrm{d}\tilde{u} \\ \mathrm{d}\tilde{v}
    \end{array}\right] = \left[\begin{array}{cc}
        \lambda b_{2} & -\lambda b_{1} \\ -\mu a_{2} & \mu a_{1}
    \end{array}\right]\left[\begin{array}{c}
        \mathrm{d}u \\ \mathrm{d}v
    \end{array}\right]
\end{equation*}
\begin{equation*}
    \left[\begin{array}{c}
        \mathrm{d}u \\ \mathrm{d}v
    \end{array}\right] = \frac{1}{\lambda\mu\left(a_{1}b_{2}-a_{2}b_{1}\right)}\left[\begin{array}{cc}
        \mu a_{1} & \lambda b_{1} \\ \mu a_{2} & \lambda b_{2}
    \end{array}\right]\left[\begin{array}{c}
        \mathrm{d}\tilde{u} \\ \mathrm{d}\tilde{v}
    \end{array}\right]
\end{equation*}
\begin{equation*}
    \begin{aligned}
        \mathbf{r}_{,\tilde{u}} &= \mathbf{r}_{,u}\frac{\partial u}{\partial\tilde{u}} + \mathbf{r}_{,v}\frac{\partial v}{\partial\tilde{u}}
        =\frac{1}{\lambda\left(a_{1}b_{2}-a_{2}b_{1}\right)}\left(a_{1}\mathbf{r}_{,u}+a_{2}\mathbf{r}_{,v}\right) // \mathbf{e}_{1} \\
        \mathbf{r}_{,\tilde{v}} &= \mathbf{r}_{,u}\frac{\partial u}{\partial\tilde{v}} + \mathbf{r}_{,v}\frac{\partial v}{\partial\tilde{v}}
        =\frac{1}{\mu\left(a_{1}b_{2}-a_{2}b_{1}\right)}\left(b_{1}\mathbf{r}_{,u}+b_{2}\mathbf{r}_{,v}\right) // \mathbf{e}_{2}
    \end{aligned}
\end{equation*}
即$\mathbf{r}_{,\tilde{u}} \perp \mathbf{r}_{,\tilde{v}}$,得证。

保长对应

保角对应

    \end{document}