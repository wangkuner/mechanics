\documentclass[12pt]{article}
    % fonts
    \usepackage{ctex}
    \usepackage{fontspec}%set fonts
    % \usepackage{mathptmx}
    \newfontfamily\cnew{Courier New}
    \usepackage{amsmath}
    \newtheorem{definition}{Def}[section]
    % \usepackage[notext]{stix}
    \usepackage{newtxmath}
    \setmainfont{Times New Roman}

    % paper set
    \usepackage{geometry}%set paper
    \geometry{
        a4paper,
        total={170mm,257mm},
        left=20mm,
        top=20mm,
    }
    % set paragraph format
    %\setlength{\baselineskip}{1.0em}
    \renewcommand{\baselinestretch}{1.0}
    % \usepackage{booktabs} % for three-line table
    \usepackage{array}
    \usepackage[yyyymmdd]{datetime}
    \renewcommand{\dateseparator}{.}

    \title{Learning Notes of Apr 2019}
    \author{Kun Wang}
    \date{2019.11.30-\today}

    \begin{document}
        \maketitle

        \section{Lie Group and Lie Algebra}

        Algebraic set(Variety): set of solutions to an algebraic equation or system of algebraic equations.
        The set of zero point of a polynomial is 1 dimension less than the affine space.

        \begin{definition}
         Group: Set $G$ with a group operation satisfying 4 axioms.
        \begin{enumerate}
           \item closure:$\forall g_1,g_2\in G,g_1g_2\in G$
           \item associativity:$\forall g_1,g_2,g_3\in G,(g_1g_2)g_3=g_1(g_2g_3)$
           \item identity:$\exists e\in G,\forall g\in G, ge=eg=g$
           \item invertibility:$\forall g\in G,\exists g^{-1}\in G, gg^{-1}=g^{-1}g=e$
        \end{enumerate}
        \end{definition}
        Commutative group/Abel group: $\forall g_{1},g_{2}\in G,g_{1}g_{2}=g_{2}g_{1}$.
        
        Lie group: groups with continuty.

        Examples:
        \begin{enumerate}
            \item $\mathbb{R}^{n}$
            \item Complex numbers of unit modulus
            \item Hamilton's quaternions: $q=a+bi+cj+dk$, $i^{2}=j^{2}=k^{2}=-1,\ ijk=-1$, 
            Conjugate of $q$ is $\bar{q}=a-bi-cj-dk$, $G=\{q|\bar{q}q=1\}$. 
            group opration is quaternionic multiplication.
            \item General linear group $GL(n,\mathbb{R})$, 
            $n$ means the invertible square matrice is of $n$ dimensions. 
            Group operation is matrix multiplication.
            \item Special linear group $SL(n)$, $G=\{A|\det(A)=1\}$. Its group manifold is of $n^{2}-1$ dimensions.
            \item Orthogonal group $O(n)$: preserve the positive definite bilinear form.
            \item Symplectic group $Sp(2n,\mathbb{R})$: preverse the bilineat anti-symmetric form.
            \item Unitary group $U(n)$: preserve the Hermitian form. The transformation keep the complex scalars unchangable.
        \end{enumerate}

        \begin{definition}
            Homomorphism: differential map $f:G\rightarrow H,\ \forall g_{1},g_{2}\in G,f(g_{1}g_{2})=f(g_{1})f(g_{2})$ 
        \end{definition}


        矩阵指数函数
        \begin{equation*}
            \begin{aligned}
                e^{At}&=I+At+\frac{(At)^2}{2!}+\frac{(At)^3}{3!}+\dots+\frac{(At)^n}{n!}+\dots\\
                \frac{d}{dt}(e^{At})&=A+A^2t+\frac{A^3t^2}{2!}+\dots+\frac{A^nt^{n-1}}{(n-1)!}+\dots\\
                &=A(I+At+\frac{(At)^2}{2!}+\dots+\frac{(At)^{n-1}}{(n-1)!}+\dots)=Ae^{At}
            \end{aligned}
        \end{equation*}
        \indent Orthogonal group: $\left\{M\in \mathbb{R}^{n\times n}|MM^T=I\right\}$, $det(M)=\pm 1$\\
        \indent Special Orthogonal group
        $SO(3)=\left\{\mathbf{R}\in\mathbb{R}^{3 \times 3}|
        \mathbf{R}^T\mathbf{R}=\mathbf{I},det(\mathbf{R})=1\right\}$
        
        \indent Special Euclidean group
        $SE(3)=\left\{\mathbf{T}=\begin{bmatrix}
            \mathbf{R} & \mathbf{x}\\
            \mathbf{0}^T & 1
        \end{bmatrix}\in\mathbb{R}^{4 \times 4}|\mathbf{R}\in SO(3),\mathbf{x}\in \mathbb{R}^{3}\right\}$
        
        \indent Lie Algebra
        $$so(3)=\{\phi\in \mathbb{R}^3\}$$
        $$se(3)=\{\xi\in \mathbb{R}^6\}$$

        把物体的所有状态定义成一个集合。
    \end{document}