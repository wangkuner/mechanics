\documentclass[12pt]{article}
    % fonts
    \usepackage{ctex}
    \usepackage{fontspec}%set fonts
    % \usepackage{mathptmx}
    \newfontfamily\cnew{Courier New}
    \usepackage{amsmath}
    % \usepackage[notext]{stix}
    \usepackage{newtxmath}
    \setmainfont{Times New Roman}

    % paper set
    \usepackage{geometry}%set paper
    \geometry{
        a4paper,
        total={170mm,257mm},
        left=20mm,
        top=20mm,
    }
    % set paragraph format
    %\setlength{\baselineskip}{1.0em}
    \renewcommand{\baselinestretch}{1.0}
    % \usepackage{booktabs} % for three-line table
    \usepackage{array}

    \title{Learning Notes of Apr 2019}
    \author{Kun Wang}
    \date{2019年4月}

    \begin{document}
        \maketitle

        \begin{center}
            \begin{tabular}{|m{16cm}|}
                \hline
                stage 1\\
                \hline
                \begin{enumerate}
                    \item The model of the satellite with solar panels.
                    \item The model of the prismatic deployable trusses.
                    \item The dynamics of the prismatic deployable trusses.
                \end{enumerate}\\
                \hline
            \end{tabular}
        \end{center}

        \section{Lie Group and Lie Algebra}
        Group: Set $G$ with composition rule satisfying 4 axioms.
        \begin{enumerate}
           \item closure:$\forall g_1,g_2\in G,g_1g_2\in G$
           \item associativity:$\forall g_1,g_2,g_3\in G,(g_1g_2)g_3=g_1(g_2g_3)$
           \item identity:$\exists e\in G,\forall g\in G, ge=eg=g$
           \item invertibility:$\forall g\in G,\exists g^{-1}\in G, gg^{-1}=g^{-1}g=e$
        \end{enumerate}
        Lie group: groups with continuty.\\ 
        \indent 矩阵指数函数
        \begin{equation*}
            \begin{aligned}
                e^{At}&=I+At+\frac{(At)^2}{2!}+\frac{(At)^3}{3!}+\dots+\frac{(At)^n}{n!}+\dots\\
                \frac{d}{dt}(e^{At})&=A+A^2t+\frac{A^3t^2}{2!}+\dots+\frac{A^nt^{n-1}}{(n-1)!}+\dots\\
                &=A(I+At+\frac{(At)^2}{2!}+\dots+\frac{(At)^{n-1}}{(n-1)!}+\dots)=Ae^{At}
            \end{aligned}
        \end{equation*}
        \indent Orthogonal group: $\left\{M\in \mathbb{R}^{n\times n}|MM^T=I\right\}$, $det(M)=\pm 1$\\
        \indent Special Orthogonal group
        $SO(3)=\left\{\mathbf{R}\in\mathbb{R}^{3 \times 3}|
        \mathbf{R}^T\mathbf{R}=\mathbf{I},det(\mathbf{R})=1\right\}$
        
        \indent Special Euclidean group
        $SE(3)=\left\{\mathbf{T}=\begin{bmatrix}
            \mathbf{R} & \mathbf{x}\\
            \mathbf{0}^T & 1
        \end{bmatrix}\in\mathbb{R}^{4 \times 4}|\mathbf{R}\in SO(3),\mathbf{x}\in \mathbb{R}^{3}\right\}$
        
        \indent Lie Algebra
        $$so(3)=\{\phi\in \mathbb{R}^3\}$$
        $$se(3)=\{\xi\in \mathbb{R}^6\}$$

        把物体的所有状态定义成一个集合。
    \end{document}