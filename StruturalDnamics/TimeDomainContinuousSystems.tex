\chapter{Time-Domain Analysis of Continuous Systems}

\section{Basis of partial differential equations}\label{sec::basis}

常系数二阶偏微分方程
\begin{equation}
    a\frac{\partial^{2}f}{\partial x^{2}}+b\frac{\partial^{2}f}{\partial x\partial y}+c\frac{\partial^{2}f}{\partial y^{2}}=0
\end{equation}
若$b^{2}-4ac>0$则为双曲线型(hyperbolic)方程;\\
若$b^{2}-4ac=0$则为抛物线型方程;\\
若$b^{2}-4ac<0$则为椭圆型方程。

连续体振动问题中的波动方程为
\begin{equation}\label{eq-1::wave}
    \frac{\partial^{2}u}{\partial x^{2}}=\frac{1}{c^{2}}\frac{\partial^{2}u}{\partial t^{2}}
\end{equation}
其中$c$为波在物质中的传播速度。
使用分离变量法可求得该微分方程的解。

令$u\left(x,t\right)=X\left(x\right)T\left(t\right)$,带入(\ref{eq-1::wave})得
$$c^{2}T(t)\frac{\partial^{2}X}{\partial x^{2}}-X(x)\frac{\partial^{2}T}{\partial t^{2}}=0$$
亦可写为
$$\frac{1}{X(x)}\frac{\partial^{2}X}{\partial x^{2}}=\frac{1}{c^{2}T(t)}\frac{\partial^{2}T}{\partial t^{2}}$$
由于等式两边变量不同,等式要成立,则等式两边必须均等于一常数,设为$-\lambda$,则可得
\begin{equation}\label{eq-1::amplitude}
    \frac{\partial^{2}X}{\partial x^{2}}+\lambda X\left(x\right)=0
\end{equation}
\begin{equation}\label{eq-1::vibration}
    \frac{\partial^{2}T}{\partial t^{2}}+c^{2}\lambda T\left(x\right)=0
\end{equation}
这是两个常系数二阶常微分方程,根据常微分方程理论,式( \ref{eq-1::amplitude} )的通解为
\begin{table}[h!]
    \begin{center}
        \caption{general solution of 2-order ode}
        \begin{tabular}{cc}
            \toprule
            $\lambda$ & $X(x)$\\
            \midrule
            $\lambda<0$ & $X(x)=A_{1}e^{\sqrt{-\lambda}x}+A_{2}e^{\sqrt{\lambda}x}$\\
            $\lambda=0$ & $X(x)=(A_{1}+A_{2}x)$\\
            $\lambda>0$ & $X(x)=A_{1}\sin\left(\sqrt{\lambda}x\right)+A_{2}\cos\left(\sqrt{\lambda}x\right)$\\
            \bottomrule
        \end{tabular}
    \end{center}
\end{table}
对于弦振动问题有约束条件$u(0,t)=0$, $u(l,t)=0$。当$\lambda\le 0$时,
方程(\ref{eq-1::wave})无非平凡解(不恒等于零的解)。故$\lambda$只能大于0,
不妨设$\lambda=\left(\omega/c\right)^2$。
则对于弦振动问题,由方程(\ref{eq-1::amplitude})及边界条件得
\begin{equation}
    \sin\left(\frac{\omega l}{c}\right)=0
\end{equation}
可得自然频率
\begin{equation}\label{eq-1::natural_frequency}
    \omega_{n}=\frac{n\pi c}{l},\ n=1,2,\cdots
\end{equation}
方程(\ref{eq-1::amplitude})的解为
\begin{equation}
    X_{n}\left(x\right)=A_{n}\sin\frac{n\pi x}{l},\ n=1,2,\cdots
\end{equation}
根据式(\ref{eq-1::vibration})(\ref{eq-1::natural_frequency})可得
\begin{equation}
    T_{n}\left(t\right)=B_{n}\sin\left(\frac{n\pi c}{l}t+\varphi_{n}\right),\ n=1,2,\cdots
\end{equation}
波动方程的解为
\begin{equation}
    u\left(x,t\right)=\sum_{n=1}^{\infty} C_{n}\sin\frac{n\pi x}{l}\sin\left(\frac{n\pi c}{l}t+\varphi_{n}\right),\ n=1,2,\cdots
\end{equation}
声波由多种单音振动组合而成,连续体的振动也是多种振动的合成,每种振动的波长为$\frac{2l}{n}$,
振动周期为$\frac{2l}{nc}$,则波速即为$c$。

\section{一维弹性波动方程}
\subsection{弦的横向自由振动}
\textbf{assumptions}: The transverse deflection is very small so that the length $l$ of the string 
and the tension $T$ are constant. The small deflection does not mean the small transverse motion 
but it meane the small deflection angle is small so $\frac{\partial^2 v}{\partial x^2}$ can be negelected.

假设弦的横向偏转很小,弦上取微元,则其转角$\theta$可近似为$\theta=\sin\theta=\tan\theta$,
在横向方向上,根据d'Alembert原理可得
\begin{equation}
    T\left(\theta+\frac{\partial \theta}{\partial x}\mathrm{d}x\right)-T\theta=\rho\mathrm{d}x\frac{\partial^{2}v}{\partial t^{2}}
\end{equation}
$\theta=\tan\theta=\frac{\partial v}{\partial x}$,进一步得
\begin{equation}
    \frac{\partial^{2}v}{\partial x^{2}}=\frac{\rho}{T}\frac{\partial^{2}v}{\partial t^{2}}
\end{equation}
记$c=\sqrt{T/\rho}$,即得式
\begin{equation}\label{eq-1::string-wave}
    \frac{\partial^{2}v}{\partial x^{2}}=\frac{1}{c^{2}}\frac{\partial^{2}v}{\partial t^{2}}
\end{equation}
弦两端加上固定约束,即$v(0,t)=0$, $v(l,t)=0$。方程(\ref{eq-1::string-wave})是可使用分离变量法求解。
根据\ref{sec::basis}节中理论,并设$\lambda=\left(\frac{\omega}{c}\right)^{2}$(对于该问题$\lambda \le 0$无非平凡解),可得
\begin{equation}
    \left\{ \begin{array}{l}
        X(x) = A\cos \frac{\omega}{c}x + B\sin \frac{\omega}{c}x\\
        T(t) = C\cos \omega t + D\sin \omega t
    \end{array}\right.
\end{equation}
将边界条件带入,得
\begin{equation}
    \left\{ \begin{array}{l}
        X(0)T(t)=AT(t)=0 \\
        X(l)T(t)=\left(A\cos \frac{\omega}{c}l + B\sin \frac{\omega}{c}l\right)T(t)=0
    \end{array}\right.
\end{equation}
由于时间的任意性,要求得非平凡解(non-trival solution),必有
\begin{equation}
    \sin \frac{\omega l}{c}=0
\end{equation}
即
\begin{equation}
    \frac{\omega l}{c}=n\pi
\end{equation}
可得固有频率(natrual frequency)
\begin{equation}
    \omega_{n}=\frac{n\pi c}{l}
\end{equation}
相应的模态振型(mode shape)为
\begin{equation}
    \phi_{n}(x)=A_{n}\sin\frac{\omega_{n}}{c}x=A_{n}\sin\frac{n\pi x}{l}
\end{equation}
弦振动的解为所有解的线性组合,即
\begin{equation}
    \begin{aligned}
    v(x,t)&=\sum_{n=1}^{\infty}\phi_{n}(x)T_{n}(t)
    =\sum_{n=1}^{\infty}A_{n}\sin\frac{n\pi x}{l}\left(C\cos \frac{n\pi c}{l} t + D\sin \frac{n\pi c}{l} t\right)\\
    &=\sum_{n=1}^{\infty}B_{n}\sin\frac{n\pi x}{l}\sin \left(\frac{n\pi c}{l}t+\alpha_{n}\right)
    \end{aligned}
\end{equation}
该解与琴弦振动发出声音由许多单音组合起来类似,每个频率为系统的固有频率,对应的振型为固有振型。

弦上点的振动会带动邻近点的振动,这种振动的传播即为波动。
弦上每一点的振动频率均相同,所以波传播频率等于振动频率。弦内弹性波的波长为$\lambda_{n}=\frac{2l}{n}$,
波传播速度为$\frac{\lambda_{n}\omega_{n}}{2\pi}=c$,故$c$为波传播速度。


\subsection{弹性杆的轴向自由振动}

取弹性杆内微元,截面处轴向内力为$P$,其惯性力为$\rho A\mathrm{d}x\frac{\partial^{2}u}{\partial x^{2}}$,
根据d'Alembert原理可得
\begin{equation}
    P+\frac{\partial P}{\partial x}\mathrm{d}x-P=\rho A\mathrm{d}x\frac{\partial^{2}u}{\partial x^{2}}
\end{equation}
根据应力应变关系$\sigma_x=E\epsilon_x$,又$P=\sigma_{x}A$,
有$P=EA\epsilon_x=EA\frac{\partial u}{\partial x}$,上式化为
\begin{equation}\label{eq-1::axial}
    \frac{\partial^{2}u}{\partial x^{2}}=\frac{\rho}{E}\frac{\partial^{2}u}{\partial x^{2}}
\end{equation}
该式与式(\ref{eq-1::string-wave})相同,波在弹性杆中传播速度为$c=\sqrt{E/\rho}$。式(\ref{eq-1::axial})可用分离变量法求解。

在弹性杆两端施加不同的约束,相同的偏微分方程可得到不同的结果。考虑如下三种边界条件
\begin{enumerate}
    \item[(1)]两端固定$u(0,t)=0$, $u(l,t)=0$;
    \item[(2)]一端固定,一端自由,自由端无外力,即无应变,$u(0,t)=0$, $\frac{\partial u}{\partial x}|(l,t)=0$;
    \item[(3)]两端自由$\frac{\partial u}{\partial x}\|(0,t)=0$, $\frac{\partial u}{\partial x}|(l,t)=0$.
\end{enumerate}

类似弦振动问题,微分方程的通解为
\begin{equation}
    u(x,t)=\left(A\cos \frac{\omega}{c}x + B\sin \frac{\omega}{c}x\right)\sin\left(\omega t + \alpha\right)
\end{equation}

\noindent 将\textbf{第一种边界条件}带入得

固有频率$$\omega_{n}=\frac{n\pi c}{l},$$

对应模态振型为$$\phi_{n}(x)=A_{n}\sin \frac{n\pi x}{l},$$

方程解为$$u(x,t)=\sum\limits_{n=1}^{\infty}B_{n}\sin\frac{n\pi x}{l}\sin \left(\frac{n\pi c}{l}t+\alpha_{n}\right)$$

\noindent 将\textbf{第二种边界条件}带入得

固有频率$$\omega_{n}=\frac{(2n-1)\pi c}{2l},$$

对应模态振型为$$\phi_{n}(x)=A_{n}\sin \frac{(2n-1)\pi x}{2l},$$

方程解为$$u(x,t)=\sum\limits_{n=1}^{\infty}B_{n}\sin\frac{(2n-1)\pi x}{2l}\sin \left(\frac{(2n-1)\pi c}{2l}t+\alpha_{n}\right)$$

\noindent 将\textbf{第三种边界条件}带入得

固有频率$$\omega_{n}=\frac{n\pi c}{l},$$

对应模态振型为$$\phi_{n}(x)=A_{n}\cos \frac{n\pi x}{l},$$

方程解为$$u(x,t)=\sum\limits_{n=1}^{\infty}B_{n}\cos\frac{n\pi x}{l}\sin \left(\frac{n\pi c}{l}t+\alpha_{n}\right)$$

从以上结果看出,固有频率和固有振型和结构参数和边界条件有关,结构得振动响应不光和结构参数和边界条件有关,还和结构初始状态(初值)有关。


\subsection{圆柱杆的自由扭转振动}

假设圆柱杆截面没有翘曲,弹性杆截面扭矩为
\begin{equation}\label{eq-1::torsion_moment}
    T=GI_{p}\frac{\partial \theta}{\partial x}
\end{equation}
其中$G$为剪切刚度,$I_{p}$为截面扭转惯量。取微元,有
\begin{equation}\label{eq-1::torsion_equilibrium}
    \left(T+\frac{\partial T}{\partial x}\mathrm{d}x\right)-T=\rho I_{p}\mathrm{d}x\frac{\partial^{2}\theta}{\partial t^{2}}
\end{equation}
将式(\ref{eq-1::torsion_moment})带入式(\ref{eq-1::torsion_equilibrium})中得
\begin{equation}\label{eq-1::torsion_vibration}
    \frac{\partial^{2}\theta}{\partial x^{2}}=\frac{\rho}{G}\frac{\partial^{2}\theta}{\partial t^{2}}
\end{equation}
则波动传播速度为$c=\sqrt{G/\rho}$。式(\ref{eq-1::torsion_vibration})与式(\ref{eq-1::axial})相同,
其解也相同。

\section{Euler-Bernoulli梁的横向自由振动}

\section{方形薄板的自由振动}

\section{模态振型的特性}

\subsection{Orthogonality}
\subsection{Scaling}
\subsection{Expasion theorm}
\subsection{Rayley quotient}