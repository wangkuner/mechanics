\chapter{Time-Domain Analysis of Continuous Systems}

\section{Basis of partial differential equations}\label{sec::basis}

常系数二阶偏微分方程
\begin{equation}
    a\frac{\partial^{2}f}{\partial x^{2}}+b\frac{\partial^{2}f}{\partial x\partial y}+c\frac{\partial^{2}f}{\partial y^{2}}=0
\end{equation}
若$b^{2}-4ac>0$则为双曲线型(hyperbolic)方程;\\
若$b^{2}-4ac=0$则为抛物线型方程;\\
若$b^{2}-4ac<0$则为椭圆型方程。

连续体振动问题中的波动方程为
\begin{equation}\label{eq-1::wave}
    \frac{\partial^{2}u}{\partial x^{2}}=\frac{1}{c^{2}}\frac{\partial^{2}u}{\partial t^{2}}
\end{equation}
其中$c$为波在物质中的传播速度。
使用分离变量法可求得该微分方程的解。

令$u\left(x,t\right)=X\left(x\right)T\left(t\right)$,带入(\ref{eq-1::wave})得
$$c^{2}T(t)\frac{\partial^{2}X}{\partial x^{2}}-X(x)\frac{\partial^{2}T}{\partial t^{2}}=0$$
亦可写为
$$\frac{1}{X(x)}\frac{\partial^{2}X}{\partial x^{2}}=\frac{1}{c^{2}T(t)}\frac{\partial^{2}T}{\partial t^{2}}$$
由于等式两边变量不同,等式要成立,则等式两边必须均等于一常数,设为$-\lambda$,则可得
\begin{equation}\label{eq-1::amplitude}
    \frac{\partial^{2}X}{\partial x^{2}}+\lambda X\left(x\right)=0
\end{equation}
\begin{equation}\label{eq-1::vibration}
    \frac{\partial^{2}T}{\partial t^{2}}+c^{2}\lambda T\left(x\right)=0
\end{equation}
这是两个常系数二阶常微分方程,根据常微分方程理论,式( \ref{eq-1::amplitude} )的通解为
\begin{table}[h!]
    \begin{center}
        \caption{general solution of 2-order ode}
        \begin{tabular}{cc}
            \toprule
            $\lambda$ & $X(x)$\\
            \midrule
            $\lambda<0$ & $X(x)=A_{1}e^{\sqrt{-\lambda}x}+A_{2}e^{\sqrt{\lambda}x}$\\
            $\lambda=0$ & $X(x)=(A_{1}+A_{2}x)$\\
            $\lambda>0$ & $X(x)=A_{1}\sin\left(\sqrt{\lambda}x\right)+A_{2}\cos\left(\sqrt{\lambda}x\right)$\\
            \bottomrule
        \end{tabular}
    \end{center}
\end{table}
对于弦振动问题有约束条件$u(0,t)=0$, $u(l,t)=0$。当$\lambda\le 0$时,
方程(\ref{eq-1::wave})无非平凡解(不恒等于零的解)。故$\lambda$只能大于0,
不妨设$\lambda=\left(\omega/c\right)^2$。
则对于弦振动问题,由方程(\ref{eq-1::amplitude})及边界条件得
\begin{equation}
    \sin\left(\frac{\omega l}{c}\right)=0
\end{equation}
可得自然频率
\begin{equation}\label{eq-1::natural_frequency}
    \omega_{n}=\frac{n\pi c}{l},\ n=1,2,\cdots
\end{equation}
方程(\ref{eq-1::amplitude})的解为
\begin{equation}
    X_{n}\left(x\right)=A_{n}\sin\frac{n\pi x}{l},\ n=1,2,\cdots
\end{equation}
根据式(\ref{eq-1::vibration})(\ref{eq-1::natural_frequency})可得
\begin{equation}
    T_{n}\left(t\right)=B_{n}\sin\left(\frac{n\pi c}{l}t+\varphi_{n}\right),\ n=1,2,\cdots
\end{equation}
波动方程的解为
\begin{equation}
    u\left(x,t\right)=\sum_{n=1}^{\infty} C_{n}\sin\frac{n\pi x}{l}\sin\left(\frac{n\pi c}{l}t+\varphi_{n}\right),\ n=1,2,\cdots
\end{equation}
声波由多种单音振动组合而成,连续体的振动也是多种振动的合成,每种振动的波长为$\frac{2l}{n}$,
振动周期为$\frac{2l}{nc}$,则波速即为$c$。

\section{一维弹性波动方程}
\subsection{弦的横向自由振动}
\textbf{assumptions}: The transverse deflection is very small so that the length $l$ of the string 
and the tension $T$ are constant. The small deflection does not mean the small transverse motion 
but it meane the small deflection angle is small so $\frac{\partial^2 v}{\partial x^2}$ can be negelected.

假设弦的横向偏转很小,弦上取微元,则其转角$\theta$可近似为$\theta=\sin\theta=\tan\theta$,
在横向方向上,根据d'Alembert原理可得
\begin{equation}
    T\left(\theta+\frac{\partial \theta}{\partial x}\mathrm{d}x\right)-T\theta=\rho\mathrm{d}x\frac{\partial^{2}v}{\partial t^{2}}
\end{equation}
$\theta=\tan\theta=\frac{\partial v}{\partial x}$,进一步得
\begin{equation}
    \frac{\partial^{2}v}{\partial x^{2}}=\frac{\rho}{T}\frac{\partial^{2}v}{\partial t^{2}}
\end{equation}
记$c=\sqrt{T/\rho}$,即得式
\begin{equation}\label{eq-1::string-wave}
    \frac{\partial^{2}v}{\partial x^{2}}=\frac{1}{c^{2}}\frac{\partial^{2}v}{\partial t^{2}}
\end{equation}
弦两端加上固定约束,即$v(0,t)=0$, $v(l,t)=0$。方程(\ref{eq-1::string-wave})是可使用分离变量法求解。
根据\ref{sec::basis}节中理论,并设$\lambda=\left(\frac{\omega}{c}\right)^{2}$(对于该问题$\lambda \le 0$无非平凡解),可得
\begin{equation}
    \left\{ \begin{array}{l}
        X(x) = A\cos \frac{\omega}{c}x + B\sin \frac{\omega}{c}x\\
        T(t) = C\cos \omega t + D\sin \omega t
    \end{array}\right.
\end{equation}
将边界条件带入,得
\begin{equation}
    \left\{ \begin{array}{l}
        X(0)T(t)=AT(t)=0 \\
        X(l)T(t)=\left(A\cos \frac{\omega}{c}l + B\sin \frac{\omega}{c}l\right)T(t)=0
    \end{array}\right.
\end{equation}
由于时间的任意性,要求得非平凡解(non-trival solution),必有
\begin{equation}
    \sin \frac{\omega l}{c}=0
\end{equation}
即
\begin{equation}
    \frac{\omega l}{c}=n\pi
\end{equation}
可得固有频率(natrual frequency)
\begin{equation}
    \omega_{n}=\frac{n\pi c}{l}
\end{equation}
相应的模态振型(mode shape)为
\begin{equation}
    \phi_{n}(x)=A_{n}\sin\frac{\omega_{n}}{c}x=A_{n}\sin\frac{n\pi x}{l}
\end{equation}
弦振动的解为所有解的线性组合,即
\begin{equation}
    \begin{aligned}
    v(x,t)&=\sum_{n=1}^{\infty}\phi_{n}(x)T_{n}(t)
    =\sum_{n=1}^{\infty}A_{n}\sin\frac{n\pi x}{l}\left(C\cos \frac{n\pi c}{l} t + D\sin \frac{n\pi c}{l} t\right)\\
    &=\sum_{n=1}^{\infty}B_{n}\sin\frac{n\pi x}{l}\sin \left(\frac{n\pi c}{l}t+\alpha_{n}\right)
    \end{aligned}
\end{equation}
该解与琴弦振动发出声音由许多单音组合起来类似,每个频率为系统的固有频率,对应的振型为固有振型。

弦上点的振动会带动邻近点的振动,这种振动的传播即为波动。
弦上每一点的振动频率均相同,所以波传播频率等于振动频率。弦内弹性波的波长为$\lambda_{n}=\frac{2l}{n}$,
波传播速度为$\frac{\lambda_{n}\omega_{n}}{2\pi}=c$,故$c$为波传播速度。


\subsection{弹性杆的轴向自由振动}

取弹性杆内微元,截面处轴向内力为$P$,其惯性力为$\rho A\mathrm{d}x\frac{\partial^{2}u}{\partial x^{2}}$,
根据d'Alembert原理可得
\begin{equation}
    P+\frac{\partial P}{\partial x}\mathrm{d}x-P=\rho A\mathrm{d}x\frac{\partial^{2}u}{\partial x^{2}}
\end{equation}
根据应力应变关系$\sigma_x=E\epsilon_x$,又$P=\sigma_{x}A$,
有$P=EA\epsilon_x=EA\frac{\partial u}{\partial x}$,上式化为
\begin{equation}\label{eq-1::axial}
    \frac{\partial^{2}u}{\partial x^{2}}=\frac{\rho}{E}\frac{\partial^{2}u}{\partial x^{2}}
\end{equation}
该式与式(\ref{eq-1::string-wave})相同,波在弹性杆中传播速度为$c=\sqrt{E/\rho}$。式(\ref{eq-1::axial})可用分离变量法求解。

在弹性杆两端施加不同的约束,相同的偏微分方程可得到不同的结果。考虑如下三种边界条件
\begin{enumerate}
    \item[(1)]两端固定$u(0,t)=0$, $u(l,t)=0$;
    \item[(2)]一端固定,一端自由,自由端无外力,即无应变,$u(0,t)=0$, $\frac{\partial u}{\partial x}|(l,t)=0$;
    \item[(3)]两端自由$\frac{\partial u}{\partial x}|(0,t)=0$, $\frac{\partial u}{\partial x}|(l,t)=0$.
\end{enumerate}

类似弦振动问题,微分方程的通解为
\begin{equation}
    u(x,t)=\left(A\cos \frac{\omega}{c}x + B\sin \frac{\omega}{c}x\right)\sin\left(\omega t + \alpha\right)
\end{equation}

\noindent 将\textbf{第一种边界条件}带入得

固有频率$$\omega_{n}=\frac{n\pi c}{l},$$

对应模态振型为$$\phi_{n}(x)=A_{n}\sin \frac{n\pi x}{l},$$

方程解为$$u(x,t)=\sum\limits_{n=1}^{\infty}B_{n}\sin\frac{n\pi x}{l}\sin \left(\frac{n\pi c}{l}t+\alpha_{n}\right)$$

\noindent 将\textbf{第二种边界条件}带入得

固有频率$$\omega_{n}=\frac{(2n-1)\pi c}{2l},$$

对应模态振型为$$\phi_{n}(x)=A_{n}\sin \frac{(2n-1)\pi x}{2l},$$

方程解为$$u(x,t)=\sum\limits_{n=1}^{\infty}B_{n}\sin\frac{(2n-1)\pi x}{2l}\sin \left(\frac{(2n-1)\pi c}{2l}t+\alpha_{n}\right)$$

\noindent 将\textbf{第三种边界条件}带入得

固有频率$$\omega_{n}=\frac{n\pi c}{l},$$

对应模态振型为$$\phi_{n}(x)=A_{n}\cos \frac{n\pi x}{l},$$

方程解为$$u(x,t)=\sum\limits_{n=1}^{\infty}B_{n}\cos\frac{n\pi x}{l}\sin \left(\frac{n\pi c}{l}t+\alpha_{n}\right)$$

从以上结果看出,固有频率和固有振型和结构参数和边界条件有关,结构得振动响应不光和结构参数和边界条件有关,还和结构初始状态(初值)有关。


\subsection{圆柱杆的自由扭转振动}

假设圆柱杆截面没有翘曲,弹性杆截面扭矩为
\begin{equation}\label{eq-1::torsion_moment}
    T=GI_{p}\frac{\partial \theta}{\partial x}
\end{equation}
其中$G$为剪切刚度,$I_{p}$为截面扭转惯量。取微元,有
\begin{equation}\label{eq-1::torsion_equilibrium}
    \left(T+\frac{\partial T}{\partial x}\mathrm{d}x\right)-T=\rho I_{p}\mathrm{d}x\frac{\partial^{2}\theta}{\partial t^{2}}
\end{equation}
将式(\ref{eq-1::torsion_moment})带入式(\ref{eq-1::torsion_equilibrium})中得
\begin{equation}\label{eq-1::torsion_vibration}
    \frac{\partial^{2}\theta}{\partial x^{2}}=\frac{\rho}{G}\frac{\partial^{2}\theta}{\partial t^{2}}
\end{equation}
则波动传播速度为$c=\sqrt{G/\rho}$。式(\ref{eq-1::torsion_vibration})与式(\ref{eq-1::axial})相同,
其解也相同。

\section{Euler-Bernoulli梁的横向自由振动}
Assumptions:
\begin{enumerate}
    \item[(1)] The cross-section is infinitely rigid in its own plane.
    \item[(2)] The cross-section of a beam remains plane after deformation.
    \item[(3)] The cross-section remains normal to the deformed neutral axis of the beam.
\end{enumerate}

Equilibrium equation of the deflection direciton
\begin{equation}\label{eq-1::beam-equil}
    -Q - \frac{\partial Q}{\partial x}\mathrm{d}x + Q - \rho A(x)\mathrm{d}x\frac{\partial^{2}v}{\partial t^{2}} = 0
\end{equation}
Constitution law
\begin{equation}\label{eq-1::beam-constitution}
    M = EI(x)\frac{\partial^{2}v}{\partial x^{2}}
\end{equation}
Equilibrium equation of moment
\begin{equation}\label{eq-1::beam-moment-1}
    Q\mathrm{d}x + M - \left(M + \frac{\partial M}{\partial x}\right) = 0
\end{equation}
The following relationship between the shear force and the moment can be derived from Eq. (\ref{eq-1::beam-moment-1}) 
\begin{equation}\label{eq-1::beam-moment-2}
    Q = \frac{\partial M}{\partial x}
\end{equation}
Substituting Eq. (\ref{eq-1::beam-constitution}) and Eq. (\ref{eq-1::beam-moment-2}) into Eq. (\ref{eq-1::beam-equil}), 
the equilibrium equation of the differential element with respect to the deflection $v$ becomes
\begin{equation}\label{eq-1::beam-equil-2}
    EI\frac{\partial ^{4} v}{\partial x^{4}} + \rho A\frac{\partial^{2}v}{\partial t^{2}} = 0
\end{equation}
Eq. (\ref{eq-1::beam-equil-2}) is a separable linear fourth-order partial differential equation. 
The form of the solution would be 
\begin{equation}\label{eq-1::form-solution}
    v(x,t)=X(x)\sin\left(\omega t + \alpha\right)
\end{equation}
Substituting Eq.(\ref{eq-1::form-solution}) into Eq.(\ref{eq-1::beam-equil-2}) 
can obtain the following ordinary differential equation 
\begin{equation}\label{eq-1::beam-ode}
    \frac{\mathrm{d}^{4} X}{\mathrm{d}x^{4}} - \frac{\rho A\omega^{2}}{EI}X(x) = 0
\end{equation}
The characteristic equation of Eq. (\ref{eq-1::beam-ode}) is 
\begin{equation}\label{eq-1::beam-character}
    \bar{\lambda}^{4} - \frac{\rho A\omega^{2}}{EI} = 0
\end{equation}
Roots for the characteristic equation are $\pm\lambda, \pm\lambda i$, with 
\begin{equation}\label{eq-1::beam-eigen}
    \lambda = \left(\frac{\rho A\omega^{2}}{EI}\right)^{\frac{1}{4}}
\end{equation}
The generalized form of solutions of Eq. (\ref{eq-1::beam-ode}) is 
\begin{equation}\label{eq-1::general-solution}
    X(x) = A_{1}e^{\lambda x} + A_{2}e^{-\lambda x} + A_{3}e^{\lambda xi} + A_{4}e^{-\lambda xi} 
\end{equation}
Based on the Euler formula, Eq. (\ref{eq-1::general-solution}) can be expressed by the linear combination 
of the trigonometric and hyperbolic functioons as 
\begin{equation}\label{eq-1::general-solution-2}
    X(x) = B_{1}\sin \lambda x + B_{2}\cos \lambda x + B_{3}\sinh \lambda x + B_{4}\cosh \lambda x 
\end{equation}

考虑三类边界条件
\begin{enumerate}
    \item[(1)] 简支,位移和弯矩为0,$v(x,t)=0$, $\frac{\partial^{2} v}{\partial x^{2}}|_{(x,t)}=0$;
    \item[(2)] 固定,位移和转角为0,$v(x,t)=0$, $\frac{\partial v}{\partial x}|_{(x,t)}=0$;
    \item[(3)] 自由,剪力和弯矩为0,$\frac{\partial^{3} v}{\partial x^{3}}|_{(x,t)}=0$, 
                $\frac{\partial^{2} v}{\partial x^{2}}|_{(x,t)}=0$.
\end{enumerate}

不同的边界条件组合,研究如下四种情形梁的自由振动。

\subsection{Simple supported beams}
boundary conditions: $v(0,t)=v(l,t)=0$, 
$\frac{\partial^{2} v}{\partial x^{2}}|_{(0,t)}=\frac{\partial^{2} v}{\partial x^{2}}|_{(l,t)}=0$. 

将边界条件带入式(\ref{eq-1::general-solution-2})得
\begin{equation}
    \begin{aligned}
        \left\{\begin{array}{l}
            B_{2} + B_{4} = 0\\
            -\lambda^{2}B_{2} + \lambda^{2}B_{4} = 0\\
            B_{1}\sin \lambda l + B_{3}\sinh \lambda l = 0\\
            -\lambda^{2}B_{1}\sin \lambda l + \lambda^{2}B_{3}\sinh \lambda l = 0
        \end{array}\right.
    \end{aligned}
\end{equation}
根据该方程组要想获得非平凡解,要求其系数矩阵行列式为0,可得
\begin{equation}
    \sin \lambda l =0
\end{equation}
即要求
\begin{equation}
    \lambda_{n} = \frac{n\pi}{l}
\end{equation}
结合式(\ref{eq-1::beam-eigen})可得固有频率
\begin{equation}
    \omega_{n}=\lambda_{n}^{2}\sqrt{\frac{EI}{\rho A}}
\end{equation}
解得$B_{2}=B_{3}=B_{4}=0$,模态振型为
\begin{equation}
    \phi_{n}(x)=A_{n}\sin\frac{n\pi x}{l}
\end{equation}

\subsection{Cantilever beams}

boundary conditions: $v(0,t)=\frac{\partial v}{\partial x}|_{(0,t)}=0$, 
$\frac{\partial^{3} v}{\partial x^{3}}|_{(l,t)}=\frac{\partial^{2} v}{\partial x^{2}}|_{(l,t)}=0$. 

将边界条件带入式(\ref{eq-1::general-solution-2})得
\begin{equation}
    \begin{aligned}
        \left\{\begin{array}{l}
            B_{2} + B_{4} = 0\\
            \lambda B_{1} + \lambda B_{3} = 0\\
            -\lambda^{2}B_{1}\sin \lambda l - \lambda^{2}B_{2}\cos \lambda l + \lambda^{2}B_{3}\sinh \lambda l + \lambda^{2}B_{4}\cosh \lambda l= 0\\
            -\lambda^{3}B_{1}\cos \lambda l + \lambda^{3}B_{2}\sin \lambda l + \lambda^{3}B_{3}\cosh \lambda l + \lambda^{3}B_{4}\sinh \lambda l= 0
        \end{array}\right.
    \end{aligned}
\end{equation}
系数矩阵行列式为0,可得
\begin{equation}
    \cos \lambda l \cosh \lambda l +1=0
\end{equation}
该方程为超越方程,需借助于数值算法求得数值解,近似解为
$\lambda_{1}l\approx1.875$, $\lambda_{2}l\approx4.694$, $\lambda_{3}l\approx 7.855$, 
$\lambda_{n}l\approx\left(n-\frac{1}{2}\right)$ when $n\ge 4$.
模态振型为
\begin{equation}
    \phi_{n}(x)=A_{n}\left[\cosh\lambda_{n}x - \cos\lambda_{n}x - 
    \frac{\cos\lambda_{n}l + \cosh\lambda_{n}l}{\sin\lambda_{n}l + \sinh\lambda_{n}l}
    \left(\sinh\lambda_{n}x - \sin\lambda_{n}x\right)\right]    
\end{equation}

\subsection{Fixed-Fixed beams}

boundary conditions: $v(0,t)=v(l,t)=0$, 
$\frac{\partial v}{\partial x}|_{(0,t)}=\frac{\partial v}{\partial x}|_{(l,t)}=0$. 

将边界条件带入式(\ref{eq-1::general-solution-2})得
\begin{equation}
    \begin{aligned}
        \left\{\begin{array}{l}
            B_{2} + B_{4} = 0\\
            \lambda B_{1} + \lambda B_{3} = 0\\
            B_{1}\sin \lambda l + B_{2}\cos \lambda l + B_{3}\sinh \lambda l + B_{4}\cosh \lambda l= 0\\
            B_{1}\cos \lambda l - B_{2}\sin \lambda l + B_{3}\cosh \lambda l + B_{4}\sinh \lambda l= 0
        \end{array}\right.
    \end{aligned}
\end{equation}
系数矩阵行列式为0,可得
\begin{equation}\label{eq-1::fixed-fixed}
    \cos \lambda l \cosh \lambda l - 1=0
\end{equation}
该方程为超越方程,需借助于数值算法求得数值解,近似解为
$\lambda_{1}l\approx 4.730$, $\lambda_{2}l\approx 7.853$,  
$\lambda_{n}l\approx \left(n + \frac{1}{2}\right)$ when $n\ge 3$.
模态振型为
\begin{equation}
    \phi_{n}(x)=A_{n}\left[\cos\lambda_{n}x - \cosh\lambda_{n}x - 
    \frac{\cos\lambda_{n}l - \cosh\lambda_{n}l}{\sin\lambda_{n}l - \sinh\lambda_{n}l}
    \left(\sinh\lambda_{n}x \sin\lambda_{n}x\right)\right]    
\end{equation}

\subsection{Free-free beams}

boundary conditions: $\frac{\partial^{3} v}{\partial x^{3}}|_{(0,t)}=\frac{\partial^{2} v}{\partial x^{2}}|_{(0,t)}=0$, 
$\frac{\partial^{3} v}{\partial x^{3}}|_{(l,t)}=\frac{\partial^{2} v}{\partial x^{2}}|_{(l,t)}=0$. 

将边界条件带入式(\ref{eq-1::general-solution-2})得
\begin{equation}
    \begin{aligned}
        \left\{\begin{array}{l}
            -\lambda^{2}B_{2} + \lambda^{2}B_{4} = 0\\
            -\lambda^{3}B_{1} + \lambda^{3} B_{3} = 0\\
            -\lambda^{2}B_{1}\sin \lambda l - \lambda^{2}B_{2}\cos \lambda l + \lambda^{2}B_{3}\sinh \lambda l + \lambda^{2}B_{4}\cosh \lambda l= 0\\
            -\lambda^{3}B_{1}\cos \lambda l + \lambda^{3}B_{2}\sin \lambda l + \lambda^{3}B_{3}\cosh \lambda l + \lambda^{3}B_{4}\sinh \lambda l= 0
        \end{array}\right.
    \end{aligned}
\end{equation}
系数矩阵行列式为0,可得
\begin{equation}
    \cos \lambda l \cosh \lambda l - 1=0
\end{equation}
该方程与式(\ref{eq-1::fixed-fixed})相同,其解也就相同。
模态振型为
\begin{equation}
    \phi_{n}(x)=A_{n}\left[\cos\lambda_{n}x + \cosh\lambda_{n}x - 
    \frac{\cos\lambda_{n}l - \cosh\lambda_{n}l}{\sin\lambda_{n}l - \sinh\lambda_{n}l}
    \left(\sin\lambda_{n}x + \sinh\lambda_{n}x\right)\right]    
\end{equation}

\section{方形薄板的自由振动}

Kinematic equations

\noindent Kirchhoff Assumptions:
\begin{enumerate}
    \item[(1)] The normal material line to the neutral plane is infinitely rigid along its own, which means that the length of the material line is constant.
    \item[(2)] The normal material line to the neutral plane remains a straight line after deformation.
    \item[(3)] The straight normal material line remains normal to the deformed neutral plane.
\end{enumerate}
根据假设(1), 垂直于中性面的应变为0,即$\epsilon_{z}=\frac{\partial w}{\partial z}=0$. 
所以位移$w$与物质坐标$z$无关。即
\begin{equation}
    w(x,y,z)=\bar{w}(x,y,z)
\end{equation}
根据假设2,面内位移可表示为
\begin{equation}
    \begin{aligned}
        u(x,y,z)=\bar{u}(x,y)+z\theta_{2}(x,y)\\
        v(x,y,z)=\bar{v}(x,y)-z\theta_{1}(x,y)
    \end{aligned}
\end{equation}
其中$\theta_{1}$, $\theta_{2}$分别为中性面法线绕$x$, $y$轴的偏转角。
根据假设(3),则有
\begin{equation}
    \theta_{1}(x,y)=\frac{\partial w}{\partial y},\ \theta_{2}=-\frac{\partial w}{\partial x}.
\end{equation}
几何方程

\begin{equation}
    \begin{aligned}
        \epsilon_{x}&=\frac{\partial u}{\partial x}=\frac{\partial \bar{u}}{\partial x}-z\frac{\partial^{2}\bar{w}}{\partial x^{2}}\\
        \epsilon_{y}&=\frac{\partial v}{\partial y}=\frac{\partial \bar{v}}{\partial y}-z\frac{\partial^{2}\bar{w}}{\partial y^{2}}\\
        \gamma_{xz}&=\frac{\partial u}{\partial z}+\frac{\partial w}{\partial x}=\frac{\partial \bar{u}}{\partial z}+\theta_{2}+\frac{\partial w}{\partial x}=0\\
        \gamma_{yz}&=\frac{\partial v}{\partial z}+\frac{\partial w}{\partial y}=\frac{\partial \bar{v}}{\partial z}-\theta_{1}+\frac{\partial w}{\partial y}=0\\
        \gamma_{xy}&=\frac{\partial u}{\partial y}+\frac{\partial v}{\partial x}=\frac{\partial \bar{u}}{\partial y}+\frac{\partial \bar{v}}{\partial x}-2z\frac{\partial^{2}\bar{w}}{\partial x\partial y}
    \end{aligned}
\end{equation}
纯弯曲时可忽略面内应变,即$\partial\bar{u}/\partial x=0$, $\partial\bar{u}/\partial y=0$, 
$\partial\bar{v}/\partial x=0$, $\partial\bar{v}/\partial y=0$。 则有
\begin{equation}\label{eq-1::geometric-equation}
    \epsilon_{x}=-z\frac{\partial^{2}\bar{w}}{\partial x^{2}},\ 
    \epsilon_{y}=-z\frac{\partial^{2}\bar{w}}{\partial y^{2}},\ 
    \gamma_{xy}=-2z\frac{\partial^{2}\bar{w}}{\partial x\partial y}
\end{equation}

本构方程
\begin{equation}
    \begin{aligned}
        \sigma_{x}&=-\frac{Ez}{1-\mu^{2}}\left(\frac{\partial^{2}\bar{w}}{\partial x^{2}}+\mu\frac{\partial^{2}\bar{w}}{\partial y^{2}}\right)\\
        \sigma_{y}&=-\frac{Ez}{1-\mu^{2}}\left(\frac{\partial^{2}\bar{w}}{\partial y^{2}}+\mu\frac{\partial^{2}\bar{w}}{\partial x^{2}}\right)\\
        \tau_{xy}&=-\frac{Ez}{1-\mu^{2}}\frac{\partial^{2}\bar{w}}{\partial x\partial y}
    \end{aligned}
\end{equation}
取微元,有
\begin{equation}\label{eq-1::force-stress}
    \begin{aligned}
        M_{x}&=\int_{-\frac{h}{2}}^{\frac{h}{2}}\sigma_{x}z\mathrm{d}z=-D\left(\frac{\partial^{2}\bar{w}}{\partial x^{2}}+\mu\frac{\partial^{2}\bar{w}}{\partial y^{2}}\right)\\
        M_{y}&=\int_{-\frac{h}{2}}^{\frac{h}{2}}\sigma_{y}z\mathrm{d}z=-D\left(\frac{\partial^{2}\bar{w}}{\partial y^{2}}+\mu\frac{\partial^{2}\bar{w}}{\partial x^{2}}\right)\\
        M_{xy}&=-(1-\mu)D\frac{\partial^{2}\bar{w}}{\partial x\partial y}
    \end{aligned}
\end{equation}
其中$$D=\frac{Eh^{3}}{12(1-\mu^{2})}$$.
对微元列平衡方程
\begin{equation}
    \begin{aligned}
        \sum F_z=0\\
        \sum M_x=0\\
        \sum M_y=0
    \end{aligned}
\end{equation}
可得
\begin{equation}
    \begin{aligned}
        \frac{\partial M_{xy}}{\partial x} + \frac{\partial M_{y}}{\partial y}&=0\\
        \frac{\partial M_{xy}}{\partial y} + \frac{\partial M_{x}}{\partial x}&=0\\
        \frac{\partial Q_{x}}{\partial x} + \frac{\partial Q_{y}}{\partial y} + \rho h\frac{\partial^{2} \bar{w}}{\partial t^{2}}&=0
    \end{aligned}
\end{equation}
将式(\ref{eq-1::force-stress})带入上式中得
\begin{equation}\label{eq-1::plate-vibration}
    D\left(\frac{\partial^{4}\bar{w}}{\partial x^{4}} + 
    2\frac{\partial^{4}\bar{w}}{\partial x^{2}\partial y^{2}} + 
    \frac{\partial^{4}\bar{w}}{\partial y^{4}}\right) + 
    \rho h\frac{\partial^{2} \bar{w}}{\partial t^{2}} =0
\end{equation}
该方程也可用分离变量法求解。

\noindent 边界条件
\begin{enumerate}
    \item[(1)] 简支边。位移和弯矩为0,即$w=0$, $M_{n}=0$。
    \item[(2)] 固定边。位移和转角为0,即$w=0$, $\partial w/\partial n=0$。
    \item[(3)] 自由边。剪力和弯矩为0,即$M_{n}=0$, $Q_{n}-\frac{\partial M_{ns}}{\partial s}=0$。
\end{enumerate}

方程(\ref{eq-1::plate-vibration})的通解形式为
\begin{equation}
    \bar{w}(x,y,t)=\bar{W}(x,y)e^{\omega t i}
\end{equation}
\begin{equation}
    \begin{aligned}
        \bar{W}(x,y)=&A_{1}\sin\alpha x\sin\beta y + A_{2}\sin\alpha x\cos\beta y + 
        A_{3}\cos\alpha x\sin\beta y + A_{4}\cos\alpha x\cos\beta y + \\
        &A_{5}\sinh\alpha x\sinh\beta y + A_{6}\sinh\alpha x\cosh\beta y + \\
        &A_{7}\cosh\alpha x\sinh\beta y + A_{8}\cosh\alpha x\cosh\beta y
    \end{aligned}
\end{equation}
根据不同的边界条件可求得特解。

对于四边简支的矩形薄板,根据边界条件可得其模态振型为
\begin{equation}
    \bar{W}(x,y)=A_{1}\sin\alpha x\sin\beta y
\end{equation}
且要满足
\begin{equation}
    \sin \alpha xa=0, \sin \beta b=0
\end{equation}
其中$a$, $b$为矩阵板的边长。
则有
\begin{equation}
    \alpha=\frac{n\pi}{a},\ \beta=\frac{m\pi}{b}
\end{equation}
其固有频率为
\begin{equation}
    \omega_{nm}=\pi^{2}\left[\left(\frac{n}{a}\right)^{2}+\left(\frac{m}{b}\right)^{2}\right]\sqrt{\frac{D}{\rho h}}
\end{equation}
其模态振型为
\begin{equation}
    \phi_{nm}=A_{nm}\sin\frac{n\pi x}{a}\sin\frac{m\pi y}{b}
\end{equation}


\section{模态振型的特性}

\subsection{Orthogonality}

\begin{equation}
    \int_{0}^{l}\rho A\phi_{r}\phi_{s}\mathrm{d}x=0,\ r\neq s.
\end{equation}

\subsection{Scaling}

modal mass
\begin{equation}\label{eq-1::model-mass}
    M_{r}=\int_{0}^{l}\rho A\bar{\phi}_{r}^{2}\mathrm{d}x=1,
\end{equation}
where $\bar{\phi}_{r}=c\phi$ is the normalized mode shape. $c$ is the scaling factor and 
\begin{equation}
    c=\sqrt{\frac{1}{\int_{0}^{l}\rho A\phi_{r}^{2}\mathrm{d}x}}
\end{equation}

modal stiffness
\begin{equation}\label{eq-1::modal-stiffness}
    K_{r}=\int_{0}^{l}EI\left(\frac{\mathrm{d}^{2}\phi_{r}}{\mathrm{d}x^{2}}\right)^{2}\mathrm{d}x
\end{equation}

\begin{equation}\label{eq-1::relation-mass-stiffness}
    \omega_{n}=\frac{K_{r}}{M_{r}}
\end{equation}

\subsection{Expasion theorm}

在前几节已经得到张紧弦,Euler-Bernoulli梁,方形薄板自由振动的模态振型。
对于任意函数$V(x)$,如果其满足如下条件
\begin{enumerate}
    \item[(1)] 与一模态振型函数集满足相同的边界条件;
    \item[(2)] $EI\left(\mathrm{d}^{4}V/\mathrm{d}x^{4}\right)$是连续函数。
\end{enumerate}
则$V(x)$可表示为收敛级数的形式。
\begin{equation}\label{eq-1::expansion}
    V(x)=\sum_{r=1}^{\infty}q_{r}\phi_{r}
\end{equation}
其中
\begin{equation}
    q_{r}=\frac{\int_{0}^{l}\rho AV(x)\phi_{r}\mathrm{d}x}{\int_{0}^{l}\rho A\phi_{r}^{2}\mathrm{d}x}
\end{equation}
亦即模态坐标。

\subsection{Rayleigh quotient}

定义Rayleigh quotient:
\begin{equation}\label{eq-1::Rayleigh-quotient}
    R(V)=\frac{\int_{0}^{l}EI\left(\frac{\mathrm{d}^{2}V}{\mathrm{d}x^{2}}\right)^{2}\mathrm{d}x}
    {\int_{0}^{l}\rho AV^{2}\mathrm{d}x}
\end{equation}
根据模态振型的正交性,模态刚度的正交性,以及式(\ref{eq-1::relation-mass-stiffness})(\ref{eq-1::expansion}),可得
\begin{equation}
    \begin{aligned}
        R(V)&=\frac{\sum_{r=1}^{\infty}q_{r}^{2}\omega_{r}^{2}}{\sum_{r=1}^{\infty}q_{r}^{2}}\\
        &=\omega_{1}^{2}\frac{1 + \sum_{r=2}^{\infty}\left(\frac{q_{r}}{q_{1}}\right)^{2}
        \left(\frac{\omega_{r}}{\omega_{1}}\right)^{2}}{1 + \sum_{r=2}^{\infty}\left(\frac{q_{r}}{q_{1}}\right)^{2}}
    \end{aligned}
\end{equation}
由于基频$\omega_{1}$最小,$\omega_{n+1}>\omega_{n}$,故
\begin{equation}\label{eq-1::upper-bound}
    R(V)\ge \omega_{1}^{2}
\end{equation}

式(\ref{eq-1::upper-bound})表示Rayleigh商是基频的上界,当$q_{r}=0,r>1$时取等号,此时$V(x)$即为基频模态振型。
如果可以选取适当的函数$V(x)$使其与基频模态振型接近,那么就可以近似估计基频的值。